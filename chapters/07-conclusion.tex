\chapter{おわりに}\label{cha:Conclusion}

本テンプレートでは、KatLab における修士論文執筆のための LaTeX 環境を提供した。

\section{本テンプレートで学んだこと}

本テンプレートでは、以下の内容を説明した:

\begin{enumerate}
  \item テンプレートの使い方と環境構築(第\ref{cha:Introduction}章)
  \item 基本的な LaTeX の書き方(第\ref{cha:Preparation}章)
  \item 図、表、数式の挿入方法(第\ref{cha:Function}章)
  \item ソースコードの挿入方法(第\ref{cha:Implementation}章)
  \item 参考文献の引用方法(第\ref{cha:Indication}章)
  \item 章立てとセクション構成(第\ref{cha:Evaluation}章)
\end{enumerate}

\section{論文執筆の流れ}

実際の論文執筆では、以下の流れで進めることを推奨する:

\begin{enumerate}
  \item 環境構築:\verb|make setup| でテンプレートをセットアップ
  \item 章構成の決定:各章のファイルを自分の研究内容に合わせて命名
  \item 内容の執筆:\verb|chapters/| ディレクトリの各ファイルを編集
  \item PDF の確認:\verb|make| で自動コンパイルしながら執筆
  \item 参考文献の追加:\verb|paper.bib| に文献情報を追加
  \item 図表の追加:\verb|images/| ディレクトリに画像を配置
  \item 最終確認:全体を通して読み、体裁を整える
\end{enumerate}

\section{便利な機能}

\subsection{自動コンパイル}

\verb|make| コマンドを実行すると、\verb|chapters/| 内のファイルの変更を自動で監視し、変更があれば自動的に PDF を再生成する。
これにより、執筆しながらリアルタイムで PDF を確認できる。

\subsection{Docker 環境}

本テンプレートは Docker を使用しているため、環境依存の問題が少なく、どのマシンでも同じ環境で執筆できる。

\section{今後の執筆に向けて}

本テンプレートを基に、自分の研究内容を執筆していく。
各章の内容を自分の研究に合わせて置き換え、図表やソースコード、参考文献を適宜追加していくことで、論文を完成させることができる。

LaTeX の詳細な使い方については、オンラインのドキュメントや書籍を参照されたい。
また、不明な点があれば、研究室のメンバーや指導教員に相談することを推奨する。
