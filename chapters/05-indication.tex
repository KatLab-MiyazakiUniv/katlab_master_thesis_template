\chapter{適用例}\label{cha:Indication}

本章では、参考文献の引用方法について説明する。

\section{参考文献の管理}

本テンプレートでは、BibTeX を使用して参考文献を管理する。
参考文献は \verb|paper.bib| ファイルに記述する。

\section{BibTeX ファイルの書き方}

\verb|paper.bib| ファイルには、以下のような形式で参考文献を記述する:

\begin{verbatim}
@article{キー名,
  author = {著者名},
  title = {タイトル},
  journal = {雑誌名},
  year = {年},
  volume = {巻数},
  pages = {ページ}
}
\end{verbatim}

\subsection{文献の種類}

以下のような文献の種類がある:

\begin{itemize}
  \item \verb|@article|: 論文
  \item \verb|@book|: 書籍
  \item \verb|@inproceedings|: 会議論文
  \item \verb|@misc|: その他(Web サイトなど)
\end{itemize}

\section{参考文献の引用}

文中で参考文献を引用するには、\verb|\cite{}| コマンドを使用する。

\subsection{引用の例}

例えば、以下のように引用できる:

\begin{itemize}
  \item 書籍を引用:LaTeX の基本的な使い方については、文献\cite{kimura-latex}を参照されたい。
  \item 学術論文を引用:Docker環境の構築手法は、文献\cite{kimura-docker}で提案されている。
  \item 会議論文を引用:自動コンパイルシステムの詳細は、文献\cite{kimura-automation}で解説されている。
  \item Webサイトを引用:テンプレートのソースコードは、文献\cite{kimura-github}で公開されている。
\end{itemize}

上記の引用は、以下のように記述する:

\begin{verbatim}
文献\cite{kimura-latex}を参照されたい。
文献\cite{kimura-docker}で提案されている。
文献\cite{kimura-automation}で解説されている。
文献\cite{kimura-github}で公開されている。
\end{verbatim}

\subsection{複数の文献を同時に引用}

複数の文献を同時に引用することもできる。
LaTeX とその周辺技術については、文献\cite{kimura-latex, kimura-docker, kimura-automation}を参照されたい。

\begin{verbatim}
文献\cite{kimura-latex, kimura-docker, kimura-automation}を参照されたい。
\end{verbatim}

\section{参考文献リストの生成}

参考文献リストは、paper.tex の最後で自動的に生成される。
引用した文献のみが参考文献リストに表示される。

\subsection{コンパイルの注意点}

参考文献を正しく表示するには、以下の順序でコンパイルが必要:

\begin{enumerate}
  \item uplatex でコンパイル
  \item bibtex で参考文献処理
  \item uplatex で再コンパイル(2回)
\end{enumerate}

本テンプレートでは、\verb|make| コマンドを使用することで、これらの処理が自動的に行われる。
